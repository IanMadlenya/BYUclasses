%% Preamble
  \documentclass{homework}

  \hwTitle{Assignment\ \#6 - Magnetic Fields in Matter} % Assignment title
  \hwDueDate{Wednesday,\ June\ 19,\ 2013} %  Due date
  \hwClass{Physics\ 441} % Course/class
  % \hwInstructor{Manuel Berrondo} % Teacher/lecturer
  \hwAuthor{Spencer Lyon} % Your name

  \usepackage{setspace}
  \usepackage{colortbl}

  %% Added by Spencer for source code highlighting
  \usepackage{listings}
  \usepackage{color}

  \definecolor{dkgreen}{rgb}{0,0.6,0}
  \definecolor{gray}{rgb}{0.5,0.5,0.5}
  \definecolor{mauve}{rgb}{0.58,0,0.82}

  \lstset{frame=tb,
    language=Python,
    aboveskip=3mm,
    belowskip=3mm,
    showstringspaces=false,
    columns=flexible,
    basicstyle={\small\ttfamily},
    numbers=left,
    stepnumber=5,
    numberstyle=\tiny\color{gray},
    keywordstyle=\color{blue},
    commentstyle=\color{dkgreen},
    stringstyle=\color{mauve},
    breaklines=true,
    breakatwhitespace=true
    tabsize=4
  }

  % Declares the font
  \usepackage{calligra}
  \DeclareMathAlphabet{\mathcalligra}{T1}{calligra}{m}{n}
  \DeclareFontShape{T1}{calligra}{m}{n}{<->s*[2.2]callig15}{}

  % Makes '\sr' make a script r
  \newcommand{\sr}{\ensuremath{\mathcalligra{r}}}

% New commands I use a lot
  \newcommand\ve{\varepsilon}
  \newcommand{\bs}[1]{\ensuremath{\boldsymbol{#1}}}
  \newcommand{\bhat}[1]{\ensuremath{\boldsymbol{\hat{#1}}}}
  \newcommand{\cross}[2]{\ensuremath{\boldsymbol{#1} \times \boldsymbol{#2}}}
  \newcommand{\curl}[1]{\ensuremath{\cross{\nabla}{\bs{#1}}}}
  \newcommand{\diver}[1]{\ensuremath{\nabla \times \bs{#1}}}

  % partial derivative as a fraction
  \newcommand{\fracpd}[2]{
    \ensuremath{\frac{\partial #1}{\partial #2}}
  }

  % partial derivative as a fraction with evaluation bounds
  \newcommand{\fracpdb}[3]{
    \ensuremath{\left. \frac{\partial #1}{\partial #2} \right|_{#3}}
  }

  % Just a vector in xhat yhat zhat
   \newcommand{\xyzvec}[3]{
   \ensuremath{
      (#1) \bhat{x} + (#2) \bhat{y} + (#3) \bhat{z}
   }
   }

  % fraction with parenthesis around it
  \newcommand{\pfrac}[2]{
    \ensuremath{ \left( \frac{#1}{#2} \right)}
  }

% Problems in this assignment
% 6.3 -> 6.3
% 6.6 -> 6.6
% 6.12 -> 6.12
% 6.23 -> 6.25
% 6.25 -> 6.23

\begin{document}

\maketitle

\begin{homeworkProblem}[Problem 6.3]

  Find the force of attraction between two magnetic dipoles, $\bs{m}_1$ and $\bs{m}_2$ oriented as shown in figure 6.7, a distance $r$ apart:

  \begin{enumerate}
    \item Using equation 6.2
    \item Using equation 6.3
  \end{enumerate}

  \vspace{.2in}

  \problemAnswer{ % Answer

    \begin{enumerate}
      \item Equation 6.2 says $$F = 2 \pi I R B \cos \theta$$ To evaluate this I need an expression for $B \cos \theta = \bs{B} \cdot \bhat{y}$. I did problem 5.34 and showed that $$ \bs{B}_{\text{dipole}} = \frac{\mu_0}{4 \pi} \frac{1}{r^3} \left[ 3(\bs{m}_1 \cdot \bhat{r}) \bhat{r}  - \bs{m}_1\right]$$ Applying that I can get the following expression:

        \begin{align*}
          \bs{B} \cdot \bhat{y} &= \frac{\mu_0}{4 \pi} \frac{1}{r^3} \left[ 3(\bs{m}_1 \cdot \bhat{r}) (\bhat{r} \cdot \bhat{y})  - (\bs{m}_1 \cdot \bhat{y}\right] \\
            &=  \frac{\mu_0}{4 \pi} \frac{1}{r^3}  3 m_1 \sin \phi \cos \phi
        \end{align*}

        The above is true because $m_1 \cdot \bhat{y} = 0$, $m_1 \bhat{r} = m1 cos \phi$, and $\bhat{r} \cdot \bhat{y} = \sin \phi$.  I now plug this in to equaiton 6.2 to get $$F = 2 \pi I R \frac{\mu_0}{4 \pi} \frac{1}{r^3} \left[ 3 m_1 \sin \phi \cos \phi \right] $$ I can simplify the trig functions here and get a final answer (note that I apply the identity that $m_2 = I R^2 \pi$ and realize that $r >> R$ to simplify a square root).

        \begin{align*}
          F &= 2 \pi I R \frac{\mu_0}{4 \pi} \frac{1}{r^3} 3 m_1 \sin \phi \cos \phi \\
            &= 2 \pi I R \frac{\mu_0}{4 \pi} \frac{1}{r^3} 3 m_1 \pfrac{R}{r} \phi \cos \phi \\
            &= 2 \pi I R \frac{\mu_0}{4 \pi} \frac{1}{r^3} 3 m_1 \pfrac{R}{r}  \pfrac{\sqrt{r^2 - R^2}}{r} \\
            &= 2 \pi I R^2 \frac{\mu_0}{4 \pi} \frac{1}{r^5} 3 m_1 \sqrt{r^2 - R^2} \\
            &= m_2 \frac{\mu_0}{2 \pi} \frac{1}{r^5} 3 m_1 \sqrt{r^2 - R^2} \\
            &= \frac{\mu_0}{2 \pi} \frac{1}{r^4} 3 m_1 m_2 \\
        \end{align*}
      \item Now I will use equation 6.3:
      \begin{align*}
        \bs{F} &= \nabla(\bs{m} \cdot \bs{B}) \\
          &= (m_2 \cdot \nabla) \bs{B} \\
          &= \left(m_2 \fracpd{}{z} \right) \frac{\mu_0}{4 \pi} \frac{1}{r^3} \left[ 3(\bs{m}_1 \cdot \bhat{r}) \bhat{r}  - \bs{m}_1\right] \\
          &= \left(m_2 \fracpd{}{z} \right) \frac{\mu_0}{4 \pi} \frac{1}{z^3} \left[ 3(\bs{m}_1 \cdot \bhat{z}) \bhat{z}  - \bs{m}_1\right] \\
          &= \left(m_2 \fracpd{}{z} \right) \frac{\mu_0}{4 \pi} \frac{1}{z^3} \left[ 2 \bs{m}_1 \right] \\
          &=  m_2 \fracpd{}{z} \frac{1}{z^3} \frac{\mu_0}{4 \pi} \left[ 2 \bs{m}_1 \right] \\
          &=  \fracpd{}{z} \frac{1}{z^3} \frac{\mu_0}{4 \pi} \left[ 2 \bs{m}_1 m_2 \right] \\
          &= \pfrac{-3}{z^4} \frac{\mu_0}{2 \pi}  m_1 m_2 \\
      \end{align*}
    \end{enumerate}
    Those are the same, so I am done. \qed
  }
\end{homeworkProblem}

\begin{homeworkProblem}[Problem 6.6]

  Of the following materials which would you expect to be paramagnetic and which diamagnetic:

  \begin{itemize}
    \item aluminium
    \item copper
    \item copper chloride ($\text{CuCl}_2$)
    \item carbon
    \item lead
    \item nitrogen($\text{N}_2$)
    \item salt ($\text{NaCl}$)
    \item sulfur
    \item water
  \end{itemize}

  \vspace{.2in}

  \problemAnswer{ % Answer

    The key to this problem is determining if each of the molecules listed has an even of odd number of electrons. If there is an even number I expect the molecule to be diagmagnetic, if there is an odd number I would expect paramagnetism. See Table~\ref{tab:magnets} for the answer.

    \qed

  }
      \begin{table}
        \begin{center}
      \setstretch{1.25}
      \scalebox{0.75}{
      \begin{tabular}{lcc}
        \hline
         \rowcolor[gray]{.7} Molecule & \# of electrons &   Magnetism \\
        \hline
        \hline
         \rowcolor[gray]{.95} Al & 13  & paramagnetic \\
         \rowcolor[gray]{.8} Cu & The book gave the answer & diamagnetic \\
         \rowcolor[gray]{.95}  $\text{CuCl}_2$ & 29 + (17 * 2) = 63 & paramagnetic \\
         \rowcolor[gray]{.8} C & 6 & diamagnetic \\
         \rowcolor[gray]{.95} Pb & 82 & diamagnetic \\
         \rowcolor[gray]{.8} $\text{N}_2$ & 14 & diamagnetic \\
         \rowcolor[gray]{.95} $\text{NaCl}$ & 11 + 17 = 28 & diamagnetic \\
         \rowcolor[gray]{.8} S & 16 & diamagnetic \\
         \rowcolor[gray]{.95} $H_2O$ & (1 *2) + 8 = 10 & diamagnetic \\
        \hline
      \end{tabular}}
      \caption{Table describing magnetism of different molecules}
      \label{tab:magnets}
      \setstretch{1.75}
    \end{center}
    \end{table}
\end{homeworkProblem}

\begin{homeworkProblem}[Problem 6.12]

  An infinitely long cylinder, of radius $R$, carries a "frozen-in" magnetization, parallel to the axis, $$\bs{M} = kx \bhat{z}$$ where $k$ is a constant and $s$ is the distance from the axis; there is no free current anywhere. FInd the magnetic field inside and outside the cylinder by two different methods:

  \begin{enumerate}
    \item As in Section 6.2, locate all the bound currents, and calculate the field they produce
    \item Use Ampere's law (in the form of equation 6.20) to find \bs{H}, and then get \bs{B} from equation 6.18 (Notice that the second method is much faster, and avoids any explicit reference to the bound currents.)
  \end{enumerate}

  \vspace{.2in}

  \problemAnswer{ % Answer

    \begin{enumerate}
      \item To do this I will need to apply equations 6.13 and 6.14 to get expressions for $\bs{J}_b$ and $\bs{K}_b$, respectively.

          \begin{align*}
            \bs{J}_b = \nabla \times \bs{M} = - k \bhat{\phi} \\
            \bs{K}_b =  \bs{M} \times \bhat{n} = k R \bhat{\phi} \\
          \end{align*}

          It is easy to show that $B = 0$ outside the surface. All you need to know is that B is in the $\bhat{z}$ direction. To find the value of $B$ inside the surface I will use equation 5.44: $$ \oint \bs{B} \cdot d \bs{l} = \mu_0 I_{\text{enc}}$$ I do this below

          \begin{align*}
            \oint \bs{B} \cdot d \bs{l} = Bl &= \mu_0 I_{\text{enc}} \\
              &= \mu_0 \left[ \int \bs{J}_b da + \bs{K}_b l \right] \\
              &= \mu_0 \left[ \int (-k \bhat{\phi}) da + (k R \bhat{\phi}) l \right] \\
              &= \mu_0 k l s \\
              B &= \mu_0 k s \bhat{z}
          \end{align*}
        \item Now I will do it the easy way using Ampere's law. I can say that \bs{H} points in the $\bhat{z}$ direction. I will use the equation 6.20 and integrate over the same loop I just used before:

        \begin{align*}
          \oint \bs{H} \cdot d \bs{l} &= I_{f_{\text{enc}}} \\
            H l &= 0 \\
            \bs{H}  &= 0
        \end{align*}

        I can then use equation 6.18 ($\bs{H} = \frac{1}{\mu_0} \bs{B} - \bs{M}$) to say that $$B = \mu_0 \bs{M}$$. I can use the same arguments as before to say that outside the surface I have $\bs{M} = 0$ and inside $\bs{M} = k x \bhat{z}$ so I get the final answer that

        $$
        \begin{cases}
          B = 0 &\text{outside} \\
          B = \mu_0 k x \bhat{z} &\text{inside}
        \end{cases}
        $$

    \end{enumerate}

    \qed

  }
\end{homeworkProblem}

\begin{homeworkProblem}[Problem 6.23]

  A familiar toy consists of donut-shaped permanent magnets (magnetization parallel to the axis), which slide frictionlessly on a vertical rod (Figure 6.31). Treat the magnets as dipoles, which mass $m_d$ and dipole moment \bs{m}.

  \begin{enumerate}
    \item If you put two back-to-back magnets on teh rod, the upper one will :float: -- the magnetic force upward balancing the gravitational force downward. At what height ($z$) does it float?
    \item If you now add a third magnet (parallel to the bottom one), what is the ratio of the two heights? (Determine the actual number to 3 significant digits)
  \end{enumerate}

  \vspace{.2in}

  \problemAnswer{ % Answer

    \begin{enumerate}
      \item I start this one using our friend in equation 5.88: the expression for the magnetic field of a dipole. I do this with $\theta =0$ to get $$\bs{B}_1 = \frac{\mu_0}{4 \pi} \frac{2m}{z^3} \bhat{z}$$. I now compute this in the direction of $\bs{m}_2$: $$\bs{m}_2 \cdot \bs{b}_1 = - \frac{\mu_0}{2 \pi} \frac{m^2}{z^3}$$ Now I need to use equation 6.3 to get an expression for \bs{F}:

        \begin{align*}
          \bs{F} &= \nabla (\bs{m} \cdot \bs{B}) \\
            &= \nabla (\bs{m}_2 \cdot \bs{B}_1) \\
            &= \fracpd{}{z} \left[ - \frac{\mu_0}{2 \pi} \frac{m^2}{z^3} \right]  \bhat{z} \\
            &= \frac{3 \mu_0}{2 \pi} \frac{m^2}{z^4} \bhat{z}
        \end{align*}

        I now need this to balance the gravitational force down ($mg$) and solve for z:

        \begin{align*}
          \bs{F} &= \bs{G} \\
            \frac{3 \mu_0}{2 \pi} \frac{m^2}{z^4} \bhat{z} &= m_d g  -\bhat{z} \\
            z &= \pfrac{3 \mu_0 m^2}{2 \pi m_d g}^{1/4} \\
        \end{align*}
      \item I can use the same expression for the force that I just derived. The only exception is that I need to replace $z$ with $z_t$ and $z_m$ for the distance between middle and top and the distance between middle and bottom, respectively. Doing this I can get expressions for the net force acting in the $z$ direction on the middle and top magnets:

      $$
      \begin{cases}
        F_{\text{net}} = 0 = \frac{3 \mu_0}{2 \pi} \frac{m^2}{z_t^4} \bhat{z} - \frac{3 \mu_0}{2 \pi} \frac{m^2}{z_m^4} \bhat{z} - m_d g \bhat{z} &\text{middle magnet} \\
        F_{\text{net}} = 0 = \frac{3 \mu_0}{2 \pi} \frac{m^2}{z_m^4} \bhat{z} - \frac{3 \mu_0}{2 \pi} \frac{m^2}{(z_m + z_t)^4} \bhat{z} - m_d g \bhat{z} &\text{top magnet}
      \end{cases}
      $$

      I can subtract these two expressions and simplify to get that $\frac{1}{z_t^4} - \frac{2}{z_m^4} + \frac{1}{(z_t + z_m)^4} = 0$. I use this to get the expression $\frac{1}{(z_t / z_m)^4} + \frac{1}{(z_t/z_m + 1)^4} = 2$. I let the computer solve this for me and I got an answer of $z_t / z_m = 0.8501$

      \qed
    \end{enumerate}

  }
\end{homeworkProblem}

\begin{homeworkProblem}[Problem 6.25]

  Notice the following parallel:

  $$
  \begin{cases}
    \nabla \cdot \bs{D} = 0, \quad\nabla \times \bs{E} = 0, \quad \ve_0 \bs{E} = \bs{D} - \bs{P} &\text{(no free charge)} \\
    \nabla \cdot \bs{B} = 0, \quad \nabla \times \bs{H} = 0, \quad \mu_0 \bs{H} = \bs{B} - \mu_0 \bs{M} &\text{(no free charge)} \\
  \end{cases}
  $$

  Thus, the transcription $\bs{D} \rightarrow \bs{B}, \bs{E} \rightarrow \bs{H}, \bs{P} \rightarrow \mu_0 \bs{M}, \ve_0 \rightarrow -\mu_0$ turn an electrostatic problem into an analogous magneto-static one. Use this, together with your knowledge of the electro-static results to re-derive:

  \begin{enumerate}
    \item The magnetic field inside a uniformly magnetized sphere
    \item The magnetic field inside a sphere of linear magnetic material in an otherwise uniform magnetic field (problem 6.18)
    \item The average magnetic field over a sphere, due to steady currents within the sphere (equation 5.93)
  \end{enumerate}

  \vspace{.2in}

  \problemAnswer{ % Answer

    \begin{enumerate}
      \item I will use equation 4.14 ($\bs{E} = - \frac{1}{3 \ve_0} \bs{P}$) to say that $$\bs{H} = - \frac{1}{3 \mu_0} (\mu_0 M) = - \frac{1}{3} M$$ I then use equation 6.18 ($\bs{H} = \frac{1}{\mu_0} \bs{B} - \bs{M}$) to say that $$\bs{B} = \mu_0 (\bs{H} + \bs{M}) = \mu_0 (- \frac{1}{3} \bs{M} + \bs{M}) = \frac{2}{3} \mu_0 \bs{M}$$
      \item I will use equation 4.49 $ \left(\bs{E} = \frac{3}{\ve_r + 2} \bs{E}_0 \rightarrow \bs{E} = \frac{1}{1 + \chi_e / 3} \bs{E}_0 \right)$ to say that $$\bs{H} = \frac{1}{1  + \chi_m / 3}\bs{H}_0$$ I then use equation 6.30 $\left(\bs{B} = \mu_0 (1 +  \chi_m) \bs{H}\right)$ and equation 6.31 $\left( \bs{B}_0 = \mu_0 \bs{H}_0 \right) $ to set up the following expression $$\frac{\bs{B}}{\mu_0 (1 + \chi_m)} = \frac{1}{(1 + \chi_m / 3)}\frac{\bs{B}_0}{\mu_0} \rightarrow \bs{B} = \pfrac{1 + \chi_m}{1 + \chi_m / 3} \bs{B}_0$$.
      \item I begin this part with the average electric field over a sphere: $\bs{E}_{\text{ave}} = - \frac{1}{4 \pi \ve_0} \frac{P}{R^3}$ I will use this in addition to equation 4.39 ($- \rho = \nabla \cdot \bs{P}$). I can use this as well as an understanding that there are no free charges to re-write the expression for the average electric field: $$\bs{E}_{\text{ave}} = - \frac{1}{4 \pi \ve_0} \frac{1}{R^3} \int \bs{P} d \tau$$ I am now in a position to make the substitutions indicated in the problem description to obtain the following: $$\bs{H}_{\text{ave}} = \frac{1}{4 \pi \mu_0} \frac{1}{R^3} \int \mu_0 \bs{M} d \tau = - \frac{1}{4 \pi R^3} \bs{m}$$ I again return to using equation 6.18 (given above) to say that

        \begin{align*}
          \bs{B}_{\text{ave}} &= -\frac{\mu_0 m}{4 \pi R^3} + \mu_0 \bs{M}_{\text{ave}} \\
            &= -\frac{\mu_0 m}{4 \pi R^3} + \mu_0 \frac{m}{4/3 \pi R^3} \\
            &=\frac{2 \mu_0 m}{4 \pi R^3}
        \end{align*}
        This is the same as equation 5.93, so I am done.

    \end{enumerate}

    \qed
  }
\end{homeworkProblem}

\end{document}
