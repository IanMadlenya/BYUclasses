%% Preamble
  \documentclass{homework}

  \hwTitle{Assignment\ \#4 - Potentials} % Assignment title
  \hwDueDate{Wednesday,\ June\ 5,\ 2013} % Due date
  \hwClass{Physics\ 441} % Course/class
  % \hwInstructor{Manuel Berrondo} % Teacher/lecturer
  \hwAuthor{Spencer Lyon} % Your name

  \usepackage{setspace}

  %% Added by Spencer for source code highlighting
  \usepackage{listings}
  \usepackage{color}

  \definecolor{dkgreen}{rgb}{0,0.6,0}
  \definecolor{gray}{rgb}{0.5,0.5,0.5}
  \definecolor{mauve}{rgb}{0.58,0,0.82}

  \lstset{frame=tb,
    language=Python,
    aboveskip=3mm,
    belowskip=3mm,
    showstringspaces=false,
    columns=flexible,
    basicstyle={\small\ttfamily},
    numbers=left,
    stepnumber=5,
    numberstyle=\tiny\color{gray},
    keywordstyle=\color{blue},
    commentstyle=\color{dkgreen},
    stringstyle=\color{mauve},
    breaklines=true,
    breakatwhitespace=true
    tabsize=4
  }

  % Declares the font
  \usepackage{calligra}
  \DeclareMathAlphabet{\mathcalligra}{T1}{calligra}{m}{n}
  \DeclareFontShape{T1}{calligra}{m}{n}{<->s*[2.2]callig15}{}

  % Makes '\sr' make a script r
  \newcommand{\sr}{\ensuremath{\mathcalligra{r}}}

% Problems in this assignment
% 3.3 -> 3.3
% 3.10 -> 3.9
% 3.14 -> 3.13
% 3.19 -> 3.18
% 3.24 -> 3.23
% 3.29 -> 3.27
% 3.31 -> 3.29


% New commands I use a lot
  \newcommand\ve{\varepsilon}
  \newcommand{\bs}[1]{\ensuremath{\boldsymbol{#1}}}
  \newcommand{\bhat}[1]{\ensuremath{\boldsymbol{\hat{#1}}}}
  \newcommand{\cross}[2]{\ensuremath{\boldsymbol{#1} \times \boldsymbol{#2}}}
  \newcommand{\curl}[1]{\ensuremath{\cross{\nabla}{\bs{#1}}}}
  \newcommand{\diver}[1]{\ensuremath{\nabla \times \bs{#1}}}

  % partial derivative as a fraction
  \newcommand{\fracpd}[2]{
    \ensuremath{\frac{\partial #1}{\partial #2}}
  }

  % partial derivative as a fraction with evaluation bounds
  \newcommand{\fracpdb}[3]{
    \ensuremath{\left. \frac{\partial #1}{\partial #2} \right|_{#3}}
  }

  % Just a vector in xhat yhat zhat
   \newcommand{\xyzvec}[3]{
   \ensuremath{
      (#1) \bhat{x} + (#2) \bhat{y} + (#3) \bhat{z}
   }
   }


  % fraction with parenthesis around it
  \newcommand{\pfrac}[2]{
    \ensuremath{ \left( \frac{#1}{#2} \right)}
  }

\begin{document}

\maketitle

\begin{homeworkProblem}[Problem 3.3]

  Find the general solution to Laplace's equation in spherical coordinates, for the case where $V$ depends only on $r$. Do the same for cylindrical coordinates, assuming $V$ depends only on $s$

  \vspace{.2in}

  \problemAnswer{ % Answer

  Laplace's equation in spherical coordinates takes the following form:

  \begin{align*}
  \nabla^2 V = \frac{1}{r^2}\frac{\partial}{\partial r} \left(r^2 \frac{\partial V}{\partial r}\right) + \frac{1}{r^2 \sin\theta} \frac{\partial}{\partial \theta} \left(\sin\theta \frac{\partial V}{\partial \theta}\right) + \frac{1}{r^2 \sin^2\theta} \frac{\partial^2 V}{\partial \varphi^2} =0
  \end{align*}

  The problem says that $V$ only depends on $r$, so we can simplify this to parts that only contain derivatives in $r$. In this case, Laplace's equation becomes:

  \begin{align*}
    \nabla^2 V = \frac{1}{r^2}\frac{\partial}{\partial r} \left(r^2 \frac{\partial V}{\partial r}\right) = 0
  \end{align*}

  We can cancel out the leading $\frac{1}{r^2}$ (because the left hand side is 0) and get $$\frac{\partial}{\partial r} \left(r^2 \frac{\partial V}{\partial r}\right) = 0$$ which simplifies down to $$\left(r^2 \frac{\partial V}{\partial r}\right) = c \rightarrow \frac{\partial V}{\partial r}= \frac{c}{r^2} \rightarrow V = - \frac{c}{r} + k$$

  Laplace's equation in cylindrical coordinates is

  \begin{align*}
    \nabla^2 V=\frac{1}{s} \frac{\partial}{\partial s} \left( s \frac{\partial V}{\partial s} \right) + \frac{1}{s^2} \frac{\partial^2 V}{\partial \phi^2} + \frac{\partial^2 V}{\partial z^2} =0
  \end{align*}

  Doing the same and keeping only terms with derivatives in $s$ we simplify to $$\nabla^2 V=\frac{1}{s} \frac{\partial}{\partial s} \left( s \frac{\partial V}{\partial s} \right) = 0$$

  I again multiply both sides by $s$ to arrive at the following expression, which I simplify to get the final result:

  \begin{align*}
    \frac{\partial}{\partial s} \left( s \frac{\partial V}{\partial s} \right) = 0 \rightarrow  s \frac{\partial V}{\partial s} = c \rightarrow \frac{\partial V}{\partial s}  = \frac{c}{s} \rightarrow V = c \ln s + k
  \end{align*}

  \qed

  }
\end{homeworkProblem}

\begin{homeworkProblem}[Problem 3.10]

  A uniform line charge $\lambda$ is placed on an infinite straight wire, a distance $d$ above a grounded conducting plane. (Let's say the wire runs parallel to the $x$-axis and directly above it, and the conducting plane is the $xy$ plane.)

  \begin{enumerate}
    \item Find the potential in the region above the plane [HINT: refer to problem 2.52] % 2.47 in manual
    \item Find the charge density $\sigma$ induced on the conducting plane
  \end{enumerate}

  \vspace{.2in}

  \problemAnswer{ % Answer

  \begin{enumerate}
    \item We know that energy (potential in parenthesis) is a function of $\bhat{r}{r^2}$ ($-\bhat{r}{r}$) in 3d and $\bhat{r}/r$ ($\ln r \bhat{r}$) in 3d. Our problem is a 2-d problem (we are dealing with a line). Using that, and remembering we pick up factor of $2 \pi \ve_0$ from integrating $E \rightarrow V$ we know that the general formula for potential in 2d is the following.

        \begin{align*}
          V(\bs{s}) &= \frac{\lambda}{2 \pi \ve_0} ln (\bs{a}/\bs{s}) \\
            &= \frac{\lambda}{4 \pi \ve_0} ln (\bs{a}^2/\bs{s}^2) \\
        \end{align*}

        In this case we say that $\bs{a} = -\bs{s}$ (it is an image problem!).  We also say that $\bs{a}$ is the distance away from the $x$-axis in the $xy$ plane. An expression for this is $\bs{a} = y + (z + d)$. We can now apply this to our expression for the potential energy to get the answer:

        \begin{align*}
          V(y, z) = \frac{\lambda}{4 \pi \ve_0} ln \left( \frac{y^2 + (z + d)^2}{y^2 + (z -d)^2} \right)
        \end{align*}
      \item Now we need to find $\sigma$ on the plane ($z=0$). We will use the equation at the top of page 126: $$\sigma = - \ve_0 \left. \fracpd{V}{z} \right|_{z=0}$$

      We apply this equation and simplify to get $\sigma(y)$ (Note I let the computer do the algebra for me and I have included the code below)

      \begin{align*}
        \sigma(y) &= - \ve_0 \left. \fracpd{V}{z} \right|_{z=0} \\
          &= \left. \frac{\lambda \left(y^{2} + \left(- d + z\right)^{2}\right) \left(\frac{\left(2 d - 2 z\right) \left(y^{2} + \left(d + z\right)^{2}\right)}{\left(y^{2} + \left(- d + z\right)^{2}\right)^{2}} + \frac{2 d + 2 z}{y^{2} + \left(- d + z\right)^{2}}\right)}{4 e_{0} \pi \left(y^{2} + \left(d + z\right)^{2}\right)} \right|_{z=0} \\
          &=  - \frac{d \lambda}{\pi \left(d^{2} + y^{2}\right)}
      \end{align*}
      % TODO: Finish from here. Almost done, just need to evaluate.
    \end{enumerate}
      \setstretch{0.6}
      \lstinputlisting[language=Python, firstline=1, lastline=14]{hw_4.py}
      \setstretch{1.5}
    \qed
  }
\end{homeworkProblem}

\begin{homeworkProblem}[Problem 3.14]

  For the infinite slot (see example 3.3), determine the charge density $\sigma(y)$ on the strip at $x=0$, assuming it is a conductor at constant potential $V_0$

  \vspace{.2in}

  \problemAnswer{ % Answer

    I am going to start with the closed form solution for the infinite slot problem using equation 3.37. This solution is $$V(x, y) = \frac{2V_0}{\pi} \tan^{-1}\left(\frac{\sin(\pi y / a)}{\sinh (\pi x / a)}\right)$$ The value of $\sigma$ is then determined by $$\sigma = - \ve_0 \fracpdb{V}{x}{x=0}$$

    Note that I used the computer code below to actually take this derivative for me.

    \begin{align*}
      \sigma &= - \ve_0 \fracpdb{V}{x}{x=0} \\
        &= -\ve_0 \fracpdb{ \frac{2V_0}{\pi} \tan^{-1}\left(\frac{\sin(\pi y / a)}{\sinh (\pi x / a)}\right)}{x}{x=0} \\
        &= 2 \frac{V_{0} \epsilon_{0} \sin{\left (\frac{\pi y}{a} \right )} \cosh{\left (\frac{\pi x}{a} \right )}}{a \left(- \cos^{2}{\left (\frac{\pi y}{a} \right )} + \cosh^{2}{\left (\frac{\pi x}{a} \right )}\right)}\\
        &= 2 \frac{V_{0} \epsilon_{0}}{a \sin{\left (\frac{\pi y}{a} \right )}}
    \end{align*}

    \setstretch{0.6}
      \lstinputlisting[language=Python, firstline=17, lastline=24]{hw_4.py}
    \setstretch{1.5}

    \qed

  }
\end{homeworkProblem}

\begin{homeworkProblem}[Problem 3.19]

  The potential at the surface of a sphere (radius $R$) is given by $$V_0 = k \cos 3 \theta$$ where $k$ is a constant. Find the potential inside and outside the sphere, as well as the surface charge density $\sigma(\theta)$ on the sphere. (assume there is not charge inside or outside the sphere.)

  \vspace{.2in}

  \problemAnswer{ % Answer

    We will start with equation 3.65, in which they give the general form for the solution to Laplace's equation in spherical coordinates:

    \begin{align}
      \label{eq:laplaceSpherical}
      V(r, \theta) = \sum_{l=0}^{\infty} \left( A_lr^l + \frac{B_l}{r^{l+1}}\right) P_l(\cos \theta)
    \end{align}

    Following example 3.6 e can say that $B_l = 0 \forall l$  because otherwise the the potential would blow up at $r=0$. With that in mind, the solution we are working with is $$V(r, \theta) = \sum_{l=0} ^{\infty} A_l r^l P_l(cos\theta)$$

    We now apply the boundary condition that $V(R) = k \cos 3 \theta$, in other words

    \begin{align}
      \label{eq:BC_3.19}
      V(R, \theta) = \sum_{l=0}^{\infty} A_l R^l P_l(cos \theta) = k cos 3 \theta
    \end{align}


    We now use the orthogonality of the Legendre polynomials to our advantage and solve for the $A_l$ that satisfy our boundary conditions.

    \begin{align*}
      \int_{-1}^1 P_l(x) P_{l'}(x) dx &= \int_0^{\pi} P_l(cos \theta) P_{l'}(cos \theta) \sin \theta d \theta \\
        &= \begin{cases} 0 & \text{ if } l' \ne l \\ \frac{2}{2l +1} & \text{ if } l' =l\end{cases}
    \end{align*}

    This means that we just need to multiply equation \ref{eq:laplaceSpherical} by the equation above and integrating to solve for $A_l$.

    \begin{align*}
      A_l &= \frac{2l + 1}{2R^l} \int_0^{\pi} V_0(\theta)P_l(\cos \theta) \sin \theta d \theta \\
        &= \frac{2l + 1}{2R^l} \int_0^{\pi}  k cos (3 \theta) P_l(\cos \theta) \sin \theta d \theta
    \end{align*}

     Now to solve for these we simply plug values of $l$ into the expression above to obtain our coefficients $A_l$. Unlike in example 3.6, our potential doesn't cause most of the $A_l$ to go away, so I will not show them here. Together with
     equation \ref{eq:laplaceSpherical}, this constitutes a complete solution for the part of the problem that asked for the potential energy.

    A lot of the work for solving for $\sigma$ has been done in the book. In Example 3.9, the authors  arrive at an expression containing $\sigma$ which I repeat below(see equation 3.83):

    $$\sum_{l=o}^{\infty} (2l + 1) A_l R^{l-1} P_l(\cos \theta) = \frac{1}{\ve_0} \sigma_0(\theta) $$

    We can simply use our expression for $A$ above and solve for $\sigma$:

    \begin{align*}
      \sigma &= \ve_0 \sum_{l=o}^{\infty} (2l + 1) A_l R^{l-1} P_l(\cos \theta) \\
        &= \frac{\ve_0}{2R} \sum_{l=o}^{\infty} (2l + 1)^2  P_l(\cos \theta) \left( \int_0^{\pi}  k cos (3 \theta) P_l(\cos \theta) \sin \theta d \theta\right)
    \end{align*}

    \qed
  }
\end{homeworkProblem}

\begin{homeworkProblem}[Problem 3.24]

  Solve Laplace's equation by separation of variables in cylindrical coordinates, assuming there is no depndence on $z$ (cylindrical symmetry). [Make sure you find all solutions to the radial equation; in particular, your results must accommodate the case of an infinite line charge, for which (of course) we already know the answer.]

  \vspace{.2in}

  \problemAnswer{ % Answer

    Laplace's equation in cylindrical coordinates is

    $$ \nabla V=\frac{1}{s} \frac{\partial}{\partial s} \left( s \frac{\partial V}{\partial s} \right) + \frac{1}{s^2} \frac{\partial^2 V}{\partial \phi^2} + \frac{\partial^2 V}{\partial z^2} =0$$

    The problem tells us we have cylindrical symmetry, no z dependence, and that simplifies Laplace's equation to

    $$ \nabla V=\frac{1}{s} \frac{\partial}{\partial s} \left( s \frac{\partial V}{\partial s} \right) + \frac{1}{s^2} \frac{\partial^2 V}{\partial \phi^2} =0$$

    We seek a separable solution of the form $V(x, \phi) = S(s) \Phi(\phi)$. If we plug this guess in and take derivatives we come to.

    $$ \frac{1}{s} \Phi \fracpd{}{s} \left(s \fracpd{S}{s} \right) + \frac{1}{s^2} S \fracpd{^2 \Phi}{\phi} = 0$$

    We now multiply though by $\frac{s^2}{S \Phi}$ and obtain:

    $$\frac{s}{S} \fracpd{}{s} \left(s \fracpd{S}{s} \right) + \frac{1}{\Phi}  \fracpd{^2 \Phi}{\phi} = 0$$

    We have successfully separated the equations because the first part has only $s$ and $S$ terms, while the second term has only $\phi$ and $\Phi$ terms. For this reason, we can say that each of those is equal to a constant.

    $$\frac{s}{S} \fracpd{}{s} \left(s \fracpd{S}{s} \right) = C_s \quad  \frac{1}{\Phi}  \fracpd{^2 \Phi}{\phi} = C_{\phi}$$

    Following the logic in example 3.3, we can say that either $C_s$ or $C_{\phi}$ is negative. If we look closely we see that we will need to make $C_{\phi}$ negative, otherwise we would get an exponential function for $Phi(\phi)$ and that doesn't make sense physically. We can now solve for $\Phi$ (again using example 3.3 as a template):

    \begin{align*}
     C_{\phi} =  -k^2 &=  \frac{1}{\Phi}  \fracpd{^2 \Phi}{\phi}  \\
      -k^2 \Phi &=  \fracpd{^2 \Phi}{\phi}\\
      &\rightarrow \Phi (\phi) = C\sin k\phi + D \cos k \phi
    \end{align*}

    Due to the periodicity of $\Phi(\phi)$, we can safely say that $k \in [0, 1, 2, \dots ]$.

    We now need to solve for $S(s)$. To do this (again following Example 3.3) we say that $C_s = k^2$. I used Mathematica to tell me that the solution to the equation is $S(s) = s^n$, where in our case  $k \in [\dots -1, 0, 1, \dots]$ (all integers).
    We can decompose this solution into a linear combination of positive and negative exponents and say that our equation takes the general form:

    $$S(s) = A s^k + Bs^{-k}$$

    Notice, however, that $S(s)$ is equal to a constant when $k=0$, so we must treat that solution a bit differently (This is what the problem note was talking about when it reminded us to find all the solutions to the radial equation). To figure out this case we set $C_s = k^2 = 0$ and solve again...

    \begin{align*}
      C_s = 0 &= \frac{s}{S} \fracpd{}{s} \left(s \fracpd{S}{s} \right) \\
      &= s \fracpd{}{s} \left(s \fracpd{S}{s} \right) \\
      E &= s \fracpd{S}{s} \\
      &\rightarrow S_{k=0} = E \ln s + F
    \end{align*}

    Notice that when $k=0$, the expression for $\Phi$ simply reduces to a constant. We will combine this constant with $F$ from above and combine into a term that we will call $a_0$ below.

    We now have everything we need to give our final answer:

    \begin{align*}
      V(s, \phi) &= S(s) \Phi(\phi) \\
        &= S_{k=0} \Phi_{k=0} + S_{k \ne 0} \Phi_{k \ne 0}\\
        &= \left( A s^k + Bs^{-k} \right) \left(C\sin k\phi + D \cos k \phi \right) + E \ln s + a_0\\
        &= a_0 + E \ln s + \sum_{k=1}^{\infty} \left( s^k [a_k \sin k \phi + b_k \cos k \phi]  + s^{-k} [c_k \sin k \phi + d_k \cos k \phi] \right)
    \end{align*}
    \qed
  }
\end{homeworkProblem}

\begin{homeworkProblem}[Problem 3.29]

  For particles (one of charge $q$, one of charge $3q$, and two of charge $-2q$) are placed at the following points:

  \begin{itemize}
    \item $q \rightarrow (0, 0, -a)$
    \item $3q \rightarrow (0, 0, a)$
    \item $-2q \rightarrow (0, -a, 0)$
    \item $-2q \rightarrow (0, a, 0)$
  \end{itemize}

  Find a simple approximate formula for the potential, valid at points far from the origin (Express the answer in spherical coordinates)

  \vspace{.2in}

  \problemAnswer{ % Answer

    We will first obtain a total charge vector by multiplying the location vectors by the charges and adding (see equation 3.100):

    \begin{align*}
      \bs{p} &= q  (0, 0, -a) + 3q  (0, 0, a) + -2q  (0, -a, 0) + -2q  (0, a, 0) \\
        &= \xyzvec{0}{2qa - 2qa}{3qa - qa} \\
        &= 2qa \bhat{z}
    \end{align*}

    Equation 3.99 tells us that the potential for a dipole is given by the following expression

    $$ V(\bs{r}) = \frac{1}{4 \pi \ve_0} \frac{\bs{p} \cdot \bs{r}}{r^2}$$

    We plug in our expression for \bs{p} and evaluate to finish. Note that we replace $\bhat{z}$ with its spherical counterpart $\bhat{z} =  \cos \theta \bhat{r} - \sin \theta \bhat{\theta}$

    \begin{align*}
      V(\bs{r}) &= \frac{1}{4 \pi \ve_0} \frac{\bs{p} \cdot \bs{r}}{r^2} \\
        &=  \frac{1}{4 \pi \ve_0} \frac{\left(2qa (  \cos \theta \bhat{r} - \sin \theta \bhat{\theta}) \right)\cdot \bs{r}}{r^2} \\
        &= \frac{1}{4 \pi \ve_0} \frac{2qa \cos \theta}{r^2}
    \end{align*}

    \qed

  }
\end{homeworkProblem}

\begin{homeworkProblem}[Problem 3.31]

  For the dipole in example 3.10, expand $1/r_{\pm}$ to order $(d/r)^3$, and use this to determine the quadropole and octopole terms in the potential.

  \vspace{.2in}

  \problemAnswer{ % Answer

    We start by recognizing that the two are separated by a distance $d$ and we follow example 3.10 to say that $r' =d / 2$. Using this info as a starting point, we will go to equation 3.94:

    \begin{align*}
      \frac{1}{\sr} &= \frac{1}{r} \sum_{n=0}^{\infty} \left( \frac{r'}{r}\right)^n P_n(\cos \alpha) \\
        &= \frac{1}{r} \sum_{n=0}^{\infty} \left( \frac{d}{2r}\right)^n P_n(\cos \alpha)
    \end{align*}

    We will follow example 3.10 and say that $\sr_+$ is the distance from $+q$ and that $\sr_-$ is the distance from $-q$. We can simply say that the expression for $\sr_-$  is the same as the one given above, with a negative angle.

    We now plug into the equation used in example 3.10:

    \begin{align*}
      V(\bs(r)) &= \frac{1}{4 \pi \ve_0} \left( \frac{q}{\sr_+} - \frac{q}{\sr_-} \right) \\
        &= \frac{1}{4 \pi \ve_0} \frac{q}{r} \sum_{n=0}^{\infty} \left( \frac{d}{2r} \right)^n [P_n (\cos \theta) - P_n (- \cos \theta)]
    \end{align*}

    We now remember that $P_n (- \cos \theta) = (-1)^n P_n(\cos \theta)$ and we can say that whenever $n$ is even the expression in the square brackets is zero, so we only keep odd terms. In other words:

    $$V(\bs{r}) = \frac{1}{4 \pi \ve_0} \frac{q}{r}  \sum_{n=\text{odd}}^{\infty} \left( \frac{d}{2r}\right)^n 2 P_n(\cos \theta) $$

    We are now ready to pull out the quadrupole and octopole terms. To do this re remember that the quadrupole terms are of order $1/r^3$ while the octopole terms are of the order $1/r^4$. To get terms of this size we simply need to take the first two terms of the series...

    \begin{align*}
      V(\bs{r}) = \frac{d^3 q \left(5 \cos ^3(\theta )-3 \cos (\theta )\right)}{32 \pi  r^4 \epsilon}+\frac{d q \cos (\theta )}{4 \pi  r^2 \epsilon } + O\left( \frac{1}{r^5} \right)
    \end{align*}

    Notice that there are not any terms with $1/r^3$ dependence so the quadrupole term is zero. The octopole term is the one containing $1/r^4$, which we can see is:

    $$V_{\text{octopole}} = \frac{d^3 q \left(5 \cos ^3(\theta )-3 \cos (\theta )\right)}{32 \pi  r^4 \epsilon}$$

    \qed



  }
\end{homeworkProblem}

\end{document}
